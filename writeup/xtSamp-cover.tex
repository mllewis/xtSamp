%basic cover letter template
\documentclass{letter}
\usepackage{hyperref}

\oddsidemargin = .2in
\evensidemargin = .2in
\textwidth = 6.3in
\topmargin = -.5in
\textheight = 9in

\newcommand {\qed}{\mbox{$\Box$}}
\renewcommand {\iff}{\Longleftrightarrow}
\newcommand {\R}{\mathbb{R}}
\newcommand {\N}{\mathbb{N}}
\newcommand {\Q}{\mathbb{Q}}
\newcommand {\Z}{\mathbb{Z}}

\newcommand {\sub}{\mbox{SB}}

\address{Molly L. Lewis\\
Department of Psychology\\
Stanford University\\
Jordan Hall\\
450 Serra Mall\\
Stanford, CA  94305-2130\\
\\
mll@stanford.edu}
\begin{document}

\begin{letter}

\opening{Editorial Board\\ 
JEP:General\\
~\\
Dear Drs. Inzlicht and Gauthier, }

Please find enclosed our submission of a Replication Report, titled ``Understanding the effect of social context on learning: A replication of Xu and Tenenbaum (2007b).'' This manuscript reports a series of five replication studies (the last preregistered on OSF at \url{https://osf.io/5xg96}) of Xu \& Tenenbaum (XT; 2007b, \emph{Dev Sci}), who examined sensitivity to the statistical sampling process underlying the evidence in a categorization task. They found evidence that participants generalized differently when a teacher showed them examples of a category than when they themselves chose the examples and the teacher subsequently labeled them. Although our experiments reveal a weaker sampling effect than the one originally reported by XT, we do find consistent evidence.

We chose to replicate XT�s work for a number of reasons. First, the original paper was an important test of an assumption that has become foundational in a large number of recent experimental and computational studies of social learning. Second, other work demonstrating this assumption has been questioned in a number of recent studies (e.g., Jenkins et al., 2015, \emph{Cognit Sci}; Spencer et al., 2011, \emph{Psych Sci}). Third, social manipulations of the sort used by XT have tended to have a lower rate of reproducibility (e.g., Open Science Collaboration, 2015), and so we were especially interested in testing the robustness of this particular effect. 

We are submitting our replication studies to JEP:General because we believe that the manuscript will be interesting not only to researchers who study categorization and social learning, but also to a broader audience interested in reproducibility issues. Our studies clearly show how repeated iteration is often necessary to fully understand an experimental effect---had we done only a single replication with high power (as estimated post-hoc from the original study), we would most likely assume that the effect was not reproducible. Instead, a set of iterations of the paradigm brought us closer to the original effect size. On the other hand, our explorations also pose a cautionary tale about identifying specific mediation relationships � though we suspect several mediators of effect size, actually testing these with sufficient power would be extremely difficult. Overall, we believe this set of studies illustrates a number of important points about replication more broadly. 

Please do not hesitate to contact us if you have any further questions or concerns. 

Sincerely,\\
\\
\\
Molly L. Lewis  and Michael C. Frank
\\
Stanford University\\

\end{letter}

\end{document}