\documentclass[man]{apa2}
\usepackage{pslatex}
\usepackage{amssymb}
\usepackage{graphicx}
\usepackage{color}
\usepackage{covington}
\usepackage[usenames,dvipsnames]{xcolor}
\usepackage{booktabs}
\usepackage{setspace}


\title{Replication of Xu and Tenenbaum 2007b}

\twoauthors{Molly L. Lewis}{Michael C. Frank}
\twoaffiliations{Department of Psychology, Stanford University}{Department of Psychology, Stanford University}


\abstract{

~\\

Keywords: communication, lexicon, language evolution}

\shorttitle{Replication of Xu and Tenenbaum 2007b}
\rightheader{Replication of Xu and Tenenbaum 2007b}

\acknowledgements{ 

~\\

\noindent Address all correspondence to Molly L. Lewis, Stanford University, Department of Psychology, Jordan Hall, 450 Serra Mall (Bldg. 420), Stanford, CA, 94305. Phone: 650-721-9270. E-mail: \texttt{mll@stanford.edu}}

\begin{document}

\maketitle              


\section{Introduction}

- failed replication attempts of related work 
- Navarro, social psych stuff less likely to replicate
- relevance to learning more generally -- all learning happens in some context. Experimental context stuff (~ME)

\section{Experiment 1}

\subsection{Methods}

\subsubsection{Participants} 
\subsubsection{Stimuli}
\footnote{All stimuli, experiments, raw data and analysis code can be found at \url{XX}. 
Analyses can be found at: \url{XX}.} 

- learning context influences the ultimate structure of the 

\subsubsection{Procedure}

\subsection{Results and Discussion}

\subsection{Conclusion}

\section{Experiment 2}

\subsection{Methods}

\subsubsection{Participants} 
\subsubsection{Stimuli}



\subsubsection{Procedure}

\subsection{Results and Discussion}

\subsection{Conclusion}



\section{Experiment 3}

\subsection{Methods}

\subsubsection{Participants} 
\subsubsection{Stimuli}


\subsubsection{Procedure}

\subsection{Results and Discussion}

\subsection{Conclusion}



\bibliographystyle{apacite2}
\bibliography{biblibrary}

\newpage
\theappendix 

\section{}

\end{document}
